\documentclass[xcolor=table]{beamer}

\usepackage{lscape, amsmath, amsfonts, amssymb, setspace, theorem, wrapfig, graphicx, float, multirow, subfig, color, rotating, multicol, datetime, natbib, venndiagram, pstricks, xkeyval, tikz, etoolbox}

\usepackage{listings}

\title{GV300 - Quantitative Political Analysis}
\subtitle{University of Essex - Department of Government}
\date{Week 6 -- 4 November, 2019}				% or you can specify a date, just write it down instead of "\today"
\author{Lorenzo Crippa} 

\usetheme[progressbar=frametitle]{metropolis}
\usecolortheme{seahorse}						% try others: wolverine; crane...


\begin{document}
\frame{
\titlepage
}

\frame{
\frametitle{Math refresher (part 1)}
\begin{enumerate}
\item Notation
	\begin{enumerate}
	\item Variables and constants
	\item Sets
	\item Operators
	\end{enumerate}
\item Linear algebra
	\begin{enumerate}
	\item Scalars
	\item Vectors
	\item Matrices
	\end{enumerate}
\item Functions
	\begin{enumerate}
	\item Basic definitions
	\item Properties
	\item Important functions
	\end{enumerate}
\end{enumerate}
}

\frame{
\frametitle{Math refresher (part 2)}
\begin{enumerate}
\item[4.] Calculus
	\begin{enumerate}
	\item[4.1] Basics
	\item[4.2] Limits
	\item[4.3] Derivative
	\item[4.4] Rules of differentiation
	\item[4.5] Extrema
	\end{enumerate}
\item[5.] Integrals
	\begin{enumerate}
	\item[5.1] Definite integral
	\item[5.2] Indefinite integral
	\item[5.3] Fundamental theorem of calculus
	\item[5.4] Rules of operation
	\end{enumerate}
\end{enumerate}
}

\frame{
\frametitle{Math Refresher - Exercises in R and Stata}
Compute the following (using either R/Stata or just pen and paper):
\begin{small}
\begin{enumerate}
\item \[
		\left[\begin{array}{ccc}
						2 	& 0 & -4\\
						1 	& 3 & 5\\
						-3 	& 1 & 4\\
						1	& 2	& 2\\
						\end{array}\right]
		\times
		\left[\begin{array}{cc}
						1 	& -4\\
						-1 	& 2\\
						3 	& 5\\
						\end{array}\right]
		\]
\item \[
		\left[\begin{array}{ccc}
						2 	& 0 & -4\\
						1 	& 3 & 5\\
						-3 	& 1 & 4\\
						\end{array}\right]
		\times
		\left[\begin{array}{ccc}
						1 	& -4	& 1\\
						-1 	& 2		& 3\\
						\end{array}\right]
		\]
\item \[
		\left[\begin{array}{ccc}
						2 	& 0 & -4\\
						1 	& 3 & 5\\
						-3 	& 1 & 4\\
						\end{array}\right]
		\times
		\left[\begin{array}{ccc}
						2 	& 1	& -3\\
						0 	& 3	& 1\\
						-4	& 5	& 4\\
						\end{array}\right]
		\]
\end{enumerate}
\end{small}
}

\frame{
\frametitle{Math Refresher - Exercises in R and Stata}
Generate a vector of 1000 random numbers (it should contain at least a 0) and call it \emph{x}. Obtain the functions below and plot them in twoway graphs. Then calculate their derivatives and plot them on a twoway graph with the original functions.
\begin{multicols}{2}
\begin{small}
\begin{enumerate}
\item Linear $$y=3x - 4$$
\item Quadratic $$y=-x^{2}+ 3x - 4$$
\item Cubic $$y=x^3-x^{2}+ 3x - 4$$
\item Logarithm $$y=ln(x) $$
\item Exponential $$y=e^{(x)} $$
\item Trigonometric $$y=sin(x)+1.3 $$
\end{enumerate}
\end{small}
\end{multicols}
}




\frame{
\frametitle{Loops, functions and programs in R and Stata}
Exercises for R users:
\begin{enumerate}
\item A function for the median and the standard deviation
	\begin{itemize}
	\item[a.] re-program a function for median and population standard deviation and call them ``median2'' and ``sd2''
	\item[b.] generate two vectors (of even and uneven length), apply the new functions to them and compare them with those obtained by applying base R's functions.
	\end{itemize}
\item Imagine you're tossing a coin n times. (X: number of heads)
	\begin{itemize}
	\item[a.] write a function that returns a data frame with all X, all possible ways to get X, and the probability of each X 
	\item[b.] apply the function to n = 10, p = 0.5. Generate a twoway plot (with X on the x-axis and p on the y-axis). 
	\item[c.] Draw 1000 observations for the number of heads obtained by tossing 10 times a fair coin. Obtain a histogram reporting the results and compare it with the plot from b.
	\end{itemize}
\end{enumerate}
}

\frame{
\frametitle{Loops, functions and programs in R and Stata}
Exercises for Stata users:
\begin{enumerate}
\item Program a function that takes as arguments: 
	\begin{itemize}
	\item[a.] the number of observations to be drawn from a random binomial distribution
	\item[b.] the number of trials
	\item[c.] the probability of success for a trial
	\end{itemize}
\item Simulate the following:
	\begin{itemize}
	\item[a.] toss a fair coin 2, 20, 200, 2000, 4000 and 8000 times and compute the mean
	\item[b.] store a boxplot and a histogram relative to the means obtained for each of the six iterations and export a pdf showing them side by side
	\item[c.] export a graph showing all iterations and all plots together (a total of 12 plots, arranged in 3 rows)
	\end{itemize}	  

\end{enumerate}
}


\frame{
\frametitle{Median, mean and population standard deviation}
{\bf Median}: ``it is the point such that as many cases are greater as are less'' (Gill 2006, 362). What if the variable is even in length?

{\bf Mean}:
$$ \mu_X = \frac{1}{N} \sum_{i=1}^{N} X_i $$

{\bf Population standard deviation}:
$$ \sigma_X = \sqrt{\frac{1}{N} \sum_{i=1}^{N} (X_i - {\mu_X})^2} $$
}

\frame{
\frametitle{The binomial distribution}
{\bf Number of possible ways each outcome can occur}:

$$ {{n}\choose{x}} = \frac{n!}{x! (n-x)!}$$

{\bf Probability that a certain outcome occurs}:
$$ P(x) = \frac{n!}{x! (n-x)!} p^x (1-p)^{n-x} $$


}

\frame{
\frametitle{Conclusion}

\begin{center}
All clear? Questions? \\
Thanks and see you next week!
\end{center}
}

\frame{
\frametitle{References}
Gill, Jeff (2006). \emph{Essential Mathematics for Political and Social Research}. Cambridge University Press.}

\end{document}
