\documentclass[xcolor=table]{beamer}

\usepackage{lscape, amsmath, amsfonts, amssymb, setspace, theorem, wrapfig, graphicx, float, multirow, subfig, color, rotating, multicol, datetime, natbib, venndiagram, pstricks, xkeyval, tikz, etoolbox, verbatim, subfig}

\usepackage[super]{nth}
\usepackage{listings}
\usepackage{xcolor}
\usepackage{natbib}
\usepackage{tikz-cd}

\definecolor{codegreen}{rgb}{0,0.6,0}
\definecolor{codegreengray}{rgb}{0,0.4,0}
\definecolor{codegray}{rgb}{0.5,0.5,0.5}
\definecolor{codeblue}{rgb}{0.00,0,0.82}
\definecolor{backcolour}{rgb}{0.95,0.95,0.92}
 
\lstdefinestyle{mystyle}{
    backgroundcolor=\color{backcolour},   
    commentstyle=\color{codegreengray},
    numberstyle=\tiny\color{codegray},
    stringstyle=\color{codegreen},    basicstyle=\ttfamily\footnotesize,
    breakatwhitespace=false,         
    breaklines=true,                 
    captionpos=b,                    
    keepspaces=true,                 
    numbers=left,                    
    numbersep=5pt,                  
    showspaces=false,                
    showstringspaces=false,
    showtabs=false,                  
    tabsize=2
}
 
\lstset{style=mystyle}

\title{GV300 - Quantitative Political Analysis}
\subtitle{University of Essex - Department of Government}
\date{Week 17 -- 20 January, 2020}				% or you can specify a date, just write it down instead of "\today"
\author{Lorenzo Crippa} 

\usetheme[progressbar=frametitle]{metropolis}
\usecolortheme{seahorse}						% try others: wolverine; crane...

\begin{document}

\frame{
\titlepage
}

\frame{
\frametitle{Today's class}
\begin{center}
Today we are all going to take part to two experiments \pause

I am going to be a participant as well as you, and Dominik will be the experimenter, working remotely via a server. 
\end{center}
}

\frame{
\frametitle{First experiment}
\begin{itemize}
\item Open \textbf{Google Chrome} and log in to Moodle \pause 
\item Hit the link for the first experiment and follow the instructions \pause
\item Always remember to hit ``Next'' to proceed! \pause
\item We will skip a comprehension quiz built into the second part
\end{itemize}
}

\frame{
\frametitle{First experiment: a few questions}
A few questions for you. Please, write down your answers. They will be used during next lecture. \pause There is no right or wrong! \pause
\begin{enumerate}
\item What is the experiment about? What is a potential research question you could answer with the experiment?
\item What is randomised in the experiment? What is the manipulation implemented? What is the treatment?
\item What are the elements in the experiment that try to exert control over the DGP? What kind of unobservables or observables may affect the causal mechanism the experiment tries to elicit?
\item Is there anything you thought was weird? Is there anything you could not understand why it is there?
\end{enumerate}
}

\frame{
\frametitle{Second experiment}
\begin{itemize}
\item Open \textbf{Google Chrome} and log in to Moodle \pause 
\item Hit the link for the first experiment and follow the instructions \pause
\item Always remember to hit ``Next'' to proceed!
\end{itemize}
}

\frame{
\frametitle{Second experiment: a few questions}
\pause
A few questions for you. Please, write down your answers. They will be used during next lecture. There is no right or wrong! \pause
\begin{enumerate}
\item What is the experiment about? What is a potential research question you could answer with the experiment?
\item What is randomised in the experiment? What is the manipulation implemented? What is the treatment?
\item What are the elements in the experiment that try to exert control over the DGP? What kind of unobservables or observables may affect the causal mechanism the experiment tries to elicit?
\item Is there anything you thought was weird? Is there anything you could not understand why it is there?
\end{enumerate}
}

\frame{
\frametitle{Conclusion}
\begin{center}
All clear? More questions? \\
Thanks and see you next week!
\end{center}
}

%\begin{frame}
%\bibliographystyle{apalike}
%\bibliography{week_16}
%\end{frame}

\end{document}